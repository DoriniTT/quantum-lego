\documentclass[11pt, a4paper]{article}

% --- UNIVERSAL PREAMBLE BLOCK ---
\usepackage[a4paper, top=2.5cm, bottom=2.5cm, left=2cm, right=2cm]{geometry}
\usepackage{fontspec}

\usepackage[english, bidi=basic, provide=*]{babel}

% Set default/Latin font to Sans Serif in the main (rm) slot
\babelfont{rm}{Noto Sans}

% Add because main language is not English (Standard practice for robustness)
\usepackage{enumitem}
\setlist[itemize]{label=-}

\usepackage{amsmath}
\usepackage{parskip}

\title{\textbf{Reference Energy of Oxygen in DFT Calculations via Experimental Water-Splitting Thermodynamics}}
\author{Methodological Report}
\date{\today}

\begin{document}

\maketitle

\section{Introduction and Context}

A well-known limitation of standard Density Functional Theory (DFT), particularly Generalized Gradient Approximation (GGA) exchange--correlation functionals such as PBE, is the inaccurate description of the oxygen molecule ($O_2$). These functionals typically overbind $O_2$ due to self-interaction errors and the difficulty of describing its triplet ground state. As a result, direct DFT calculations of $O_2$ often lead to significant systematic errors in reaction energies involving oxygen, typically on the order of $0.4$--$1.0\,\text{eV}$.

To avoid propagating this error, a common approach in computational catalysis and electrochemistry is to bypass the direct DFT energy of $O_2$ and instead define an \emph{effective reference energy} for oxygen. This reference is constructed by enforcing agreement with the experimentally measured Gibbs free energy of the water-splitting reaction. This procedure is widely used in electrocatalysis studies and is consistent with the Computational Hydrogen Electrode (CHE) framework.

The resulting quantity is not the physical electronic energy of an isolated oxygen molecule, but rather an effective reference energy that ensures thermodynamic consistency with experimental data.

\section{Thermodynamic Framework}

The reference energy of oxygen is determined by fixing the chemical potential of oxygen to reproduce the experimental Gibbs free energy of the water-splitting reaction:

\begin{equation}
2H_2O(g) \rightarrow 2H_2(g) + O_2(g)
\end{equation}

Under standard conditions ($T=298.15\,\text{K}$, $p=1\,\text{bar}$), the experimental Gibbs free energy change for this reaction is

\begin{equation}
\Delta G_{exp} = 4.92\,\text{eV}
\end{equation}

All energies in this work are expressed in electronvolts (eV) per molecule. Gas-phase species are treated within the ideal gas approximation at standard conditions.

The thermodynamic definition of the reaction energy is

\begin{equation}
\Delta G_{exp} = G(O_2) + 2G(H_2) - 2G(H_2O)
\end{equation}

from which the Gibbs free energy of oxygen can be obtained:

\begin{equation}
G(O_2) = \Delta G_{exp} + 2G(H_2O) - 2G(H_2)
\end{equation}

\section{Connection Between Gibbs Free Energy and DFT Energies}

For an isolated molecule, the Gibbs free energy is written as

\begin{equation}
G = E_{DFT} + ZPE - TS
\end{equation}

where $E_{DFT}$ is the electronic total energy obtained from a DFT calculation, $ZPE$ is the zero-point energy, and $TS$ is the entropic contribution at temperature $T$.

Substituting this relation into the previous expression yields

\begin{equation}
\begin{aligned}
G(O_2) = &\, \Delta G_{exp} + 2[E_{DFT}(H_2O) + ZPE_{H_2O} - TS_{H_2O}] \\
&- 2[E_{DFT}(H_2) + ZPE_{H_2} - TS_{H_2}]
\end{aligned}
\end{equation}

Since the goal is to define an effective electronic reference energy for oxygen, the thermodynamic corrections of oxygen must be removed:

\begin{equation}
E_{ref}(O_2) = G(O_2) - ZPE_{O_2} + TS_{O_2}
\end{equation}

Combining the expressions gives

\begin{equation}
\begin{aligned}
E_{ref}(O_2) = &\, \Delta G_{exp} + 2[E_{DFT}(H_2O) + ZPE_{H_2O} - TS_{H_2O}] \\
&- 2[E_{DFT}(H_2) + ZPE_{H_2} - TS_{H_2}] \\
&- ZPE_{O_2} + TS_{O_2}
\end{aligned}
\end{equation}

This expression defines an effective oxygen reference energy that replaces the direct DFT energy of $O_2$ in subsequent calculations.

\section{Thermodynamic Corrections and Assumptions}

Standard thermodynamic corrections for gas-phase molecules at $298.15\,\text{K}$ and $1\,\text{bar}$ are used:

\begin{itemize}
\item \textbf{$H_2O$:} $ZPE = 0.56\,\text{eV}$, $TS = 0.67\,\text{eV}$
\item \textbf{$H_2$:} $ZPE = 0.27\,\text{eV}$, $TS = 0.41\,\text{eV}$
\item \textbf{$O_2$:} $ZPE = 0.10\,\text{eV}$, $TS = 0.64\,\text{eV}$
\end{itemize}

These values are commonly adopted in electrocatalysis literature. The resulting constant depends on the chosen thermodynamic corrections and may vary slightly across studies.

This approach assumes that DFT provides sufficiently accurate energies for $H_2$ and $H_2O$, while the dominant systematic error arises from the description of $O_2$. The method therefore replaces the DFT error in the O=O bond by enforcing experimental water-splitting thermodynamics.

In surface catalysis applications, entropic contributions of adsorbed species are typically neglected, while gas-phase molecules include full thermodynamic corrections.

\section{Calculation of the Reference Energy Constant}

Substituting the numerical values gives

\begin{equation}
\begin{aligned}
E_{ref}(O_2) = &\, 2E_{DFT}(H_2O) - 2E_{DFT}(H_2) + 4.92 \\
&+ 2(0.56 - 0.67) - 2(0.27 - 0.41) - 0.10 + 0.64
\end{aligned}
\end{equation}

The constant contribution is evaluated as follows:

\begin{itemize}
\item Water contribution: $2(0.56-0.67) = -0.22\,\text{eV}$
\item Hydrogen contribution: $-2(0.27-0.41) = +0.28\,\text{eV}$
\item Oxygen contribution: $-0.10 + 0.64 = +0.54\,\text{eV}$
\end{itemize}

The total constant becomes

\begin{equation}
4.92 - 0.22 + 0.28 + 0.54 = 5.52\,\text{eV}
\end{equation}

\section{Final Formulation}

The effective oxygen reference energy is therefore

\begin{equation}
\boxed{E_{ref}(O_2) = 2E_{DFT}(H_2O) - 2E_{DFT}(H_2) + 5.52\,\text{eV}}
\end{equation}

where $E_{DFT}(H_2O)$ and $E_{DFT}(H_2)$ are the total energies of isolated gas-phase molecules obtained from DFT calculations. The constant term accounts for experimental reaction thermodynamics, zero-point energies, and entropic contributions at standard conditions.

The quantity $E_{ref}(O_2)$ should be used as the oxygen reference energy in binding or formation energy calculations instead of the direct DFT energy of $O_2$. This ensures thermodynamic consistency with experimental water-splitting energetics and reduces systematic errors associated with the DFT description of the oxygen molecule.

\end{document}

